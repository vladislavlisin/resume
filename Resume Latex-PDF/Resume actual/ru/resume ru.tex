\documentclass{resume} % Use the custom resume.cls style
\usepackage{verbatim}
\usepackage[utf8]{inputenc}
\usepackage[T2A]{fontenc}
\usepackage[russian]{babel} % Для правильной расстановки переносов слов
\usepackage[left=0.4 in,top=0.4in,right=0.4 in,bottom=0.4in]{geometry} % Document margins
\usepackage{hyperref} % Добавляем пакет для корректных ссылок

\newcommand{\tab}[1]{\hspace{.2667\textwidth}\rlap{#1}}
\newcommand{\itab}[1]{\hspace{0em}\rlap{#1}}
\name{Владислав Лисин} % Ваше имя на русском
% You can merge both of these into a single line, if you do not have a website.
\address{+7 (953) 874 27 07 \\ Новосибирск, Россия}
\address{
\href{mailto:v.lisin1@alumni.nsu.ru}{v.lisin1@alumni.nsu.ru} \\
\href{https://t.me/VladislavLisin02}{telegram} \\
\href{https://api.whatsapp.com/message/5E7KGOPUVIBPB1}{whatsapp}
}  %

\begin{document}
\large
\begin{rSection}{О СЕБЕ} % Перевод секции Summary

{\textbf{Data Scientist} с опытом решения различных задач в области анализа данных, с фокусом на финтех направлении. Понимаю процессы жизненного цикла AI моделей от этапа постановки задачи до внедрения в продакшн и постановки на мониторинг. Занимался разработкой, валидацией и мониторингом ML моделей. Владею основами риск-менеджмента и риск-моделирования.} % Перевод текста

\end{rSection}

\begin{rSection}{НАВЫКИ} % Перевод секции SKILLS

\begin{tabular}{ @{} >{\bfseries}l @{\hspace{3ex}} p{0.7\textwidth} }
Программирование & Python (numpy, pandas, scipy, matplotlib), SQL, Git % Проф термины оставлены
\\
Машинное обучение
& Classical ML stack (sklearn, catboost, lightgbm, xgboost, statsmodels, optuna)
\\
&  EDA \& Feature Engineering \& Model Evaluation% Проф термины оставлены
\\
& Feature selection \& Hyperparameters tuning (optuna) % Проф термины оставлены
\\
Статистика и теорвер & Проверка гипотез, A/B тесты % Проф термины оставлены
\\
Риск-моделирование & PD, LGD, EAD, ПВР, TTC, PIT % Проф термины оставлены
\\
Рабочий процесс & Jira, Confluence, Agile, Excel % Проф термины оставлены
\\
Soft skills & Английский (B2), Соблюдение сроков, Командная работа % Soft skills переведены
\\
\end{tabular}\\

\end{rSection}

\begin{rSection}{ОПЫТ}

\textbf{Data Scientist} \hfill Июль 2025 - н.в. \\ 
Альфа Банк, Управление риск-моделирования
 \begin{itemize}
    \itemsep -3pt {}
\item Разработка PD/LGD/EAD моделей для заемщиков СБ и КИБ;
\item Составление отчетности и BRD на внедрение разработанных моделей;
\item Разработка подходов и сценариев автоматического стресс-тестирования, создание методологической базы для автоматизированной процедуры стресс-тестирования;
\item Проведение исследований и подготовка отчётов по стресс-тестированию корпоративного портфеля;
\item Аналитика рискованности отраслевого кредитования, аналитические исследования при выработке подходов к кредитованию заемщиков блоков КИБ и СБ;
\item Разработка методик портфельного резервирования МСФО;
 \end{itemize}

stack: hadoop, spark, hive, airflow, impala, jupyter lab, 

\textbf{Data Scientist} \hfill Июнь 2023 - Июль 2025 \\ 
Сбер, Управление модельных рисков (КИБ)
 \begin{itemize}
    \itemsep -3pt {}
\item Валидация и мониторинг моделей оценки компонент кредитного риска, контроль качественной и количественной эффективности работы моделей;
\item Построение моделей кредитного риска, формирование и анализ выборок для моделирования;
\item Проведение PIT и TTC калибровок рисковых моделей;
\item Разработка конвейера автоматического моделирования white-box моделей;
\item Поиск критических мест работы моделей, оценка модельного риска и предиктивная аналитика;
\item Разработка рекомендаций по улучшению качества моделей, данных и документации.
 \end{itemize}

stack: hadoop, spark, hive, impala, jupyter lab, qlick sense, bitbacket, jira, confluence, sberds

\end{rSection}

\begin{rSection}{ОБРАЗОВАНИЕ} % Этот раздел уже был на русском, оставляем как есть

{\bf Магистр, Факультет прикладной математики и информатики} \hfill {2024 - ...} \\
Новосибирский Государственный Технический Университет (НГТУ) \\ % Добавил русское сокращение
Системы ИИ и Машинное Обучение

{\bf Бакалавр, Механико-математический факультет} \hfill {2020 - 2024} \\
Новосибирский Государственный Университет (НГУ) \\ % Добавил русское сокращение
Математика и Компьютерные науки % Небольшая корректировка формулировки

\end{rSection}

\begin{rSection}
{}
\end{rSection}

\end{document}
