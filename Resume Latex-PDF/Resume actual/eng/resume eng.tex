\documentclass{resume} % Use the custom resume.cls style
\usepackage{verbatim}
\usepackage[utf8]{inputenc}
\usepackage[T2A]{fontenc}
\usepackage[russian]{babel} % Для правильной расстановки переносов слов
\usepackage[left=0.4 in,top=0.4in,right=0.4 in,bottom=0.4in]{geometry} % Document margins

\newcommand{\tab}[1]{\hspace{.2667\textwidth}\rlap{#1}} 
\newcommand{\itab}[1]{\hspace{0em}\rlap{#1}}
\name{Vladislav Lisin} % Your name
% You can merge both of these into a single line, if you do not have a website.
\address{+7 (953) 874 27 07 \\ Novosibirsk, Russia} 
\address{
\href{mailto:v.lisin1@alumni.nsu.ru}{v.lisin1@alumni.nsu.ru} \\ 
\href{https://t.me/VladislavLisin02}{telegram} \\ 
\href{https://api.whatsapp.com/message/5E7KGOPUVIBPB1}{whatsapp}
}  %

\begin{document}
\large
\begin{rSection}{Summary}

{\textbf{Data Scientist} with experience in solving various problems in the field of data analysis, with a focus on the fintech direction. I understand the life cycle processes of AI models from the stage of setting the task to implementation in production and monitoring. I was engaged in the development, validation and monitoring of ML models. I have a basic understanding of risk management and risk modeling.}

\end{rSection}

\begin{rSection}{SKILLS}


\begin{tabular}{ @{} >{\bfseries}l @{\hspace{6ex}} l }

Programming & Python (numpy, pandas, scipy, matplotlib), SQL, Git
\\
Machine Learning &  EDA \& Feature Engineering
\\
& Classical ML stack
\\
& feature selection \& hyperparametrs tuning
\\
& sklearn, statsmodels, optuna, catboost, lightgbm, xgboost
\\
Maths & Probability theority, Statistics, A/B tests
\\
Risk Modeling & PD, LGD, EAD, ПВР, TTC, PIT
\\
Work' Process & Jira, Confluence, Agile, Excel
\\
Soft skills & English (B2), Meeting deadlines, Teamwork
\\

\end{tabular}\\

\end{rSection}

\begin{rSection}{Experience}

\textbf{Junior Data Scientist} \hfill January 2024 - now \\
Sber, Model Risk Management %\hfill \textit{San Francisco, CA}
 \begin{itemize}
    \itemsep -3pt {} 
\item Validation and monitoring of models for assessing credit risk components
\item Development of a pipeline for automatic modeling of white-box models
\item Model risk assessment and predictive analytics
\item Construction of credit risk models, formation and analysis of samples for modeling
\item Conducting calibrations of risk models
 \end{itemize}
 
\textbf{Intern Data scientist} \hfill June 2023 - January 2024\\
Sber, Validation Management %\hfill \textit{San Francisco, CA}
 \begin{itemize}
    \itemsep -3pt {}
\item Validation of models for assessing credit risk components
\item Monitoring the qualitative and quantitative effectiveness of models
\item Developing recommendations for improving the quality of models, data and documentation
\item Finding critical areas of model performance, stress testing and assessing model risk
\item Alternative modeling
 \end{itemize}

\end{rSection} 
\begin{rSection}{ОБРАЗОВАНИЕ}

{\bf Master's degree, Faculty of Applied Mathematics and Computer Science} \hfill {2024 - ...} \\
Novosibirsk State Technical University (NSTU) \\
AI Systems and Machine Learning

{\bf Bachelor's degree, Faculty of Mechanics and Mathematics} \hfill {2020 - 2024} \\
Novosibirsk State University (NSU) \\
Mathematics and Computer Science

\end{rSection}

\begin{rSection}
{} 
\end{rSection}

\end{document}
